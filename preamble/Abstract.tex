\chapter*{Resumo}
\addcontentsline{toc}{chapter}{Resumo}

A fragmentação dos sistemas de informação em ambiente hospitalar constitui um risco sistémico para a segurança do doente, criando silos de informação que comprometem a gestão do ciclo do medicamento. Este projeto aborda este problema no contexto da Santa Casa da Misericórdia de Vila Verde (SCMVV), onde sistemas legados geravam ineficiências e potenciais erros. Para resolver esta lacuna, foi desenvolvida e avaliada uma solução de software integrada. Adotando uma metodologia de \textit{Design Science Research}, o projeto seguiu uma abordagem de Investigação-Ação para criar um sistema robusto, com uma arquitetura de microsserviços (Node.js) e uma interface moderna (React), sobre uma base de dados Oracle. A implementação resultou numa redução de 75\% nos erros de medicação, uma melhoria de 60\% no tempo de resposta do sistema e uma elevada satisfação dos utilizadores, validada por uma pontuação na \textit{System Usability Scale} (SUS) entre 85 e 92, mantendo total compatibilidade com os sistemas existentes. Conclui-se que esta abordagem metodológica e tecnológica não só é eficaz na modernização de processos clínicos, como também oferece um retorno de investimento positivo em 18 meses, validando o seu valor estratégico.

\vspace{6mm}
\noindent\textbf{Palavras-chave:} Gestão Medicamentosa Hospitalar, Sistemas de Informação em Saúde, Segurança do Paciente, \textit{Design Science Research}, Microsserviços, Interoperabilidade.

\vspace*{\fill}

\chapter*{Abstract}
\addcontentsline{toc}{chapter}{Abstract}

The fragmentation of information systems in hospital environments constitutes a systemic risk to patient safety, creating information silos that compromise the medication management lifecycle. This project addresses this problem within the context of the Santa Casa da Misericórdia de Vila Verde (SCMVV), where legacy systems generated inefficiencies and potential errors. To resolve this gap, an integrated software solution was developed and evaluated. Adopting a \textit{Design Science Research} (DSR) methodology, the project was conducted through an \textit{Action Research} approach to create a robust system featuring a microservices architecture (Node.js) and a modern user interface (React), built upon an Oracle Database. The implementation resulted in a 75\% reduction in medication errors, an 80\% improvement in system response times, and a high user satisfaction score of 85 to 92, while maintaining full compatibility with existing systems. In conclusion, the methodological and technological approach is not only effective for modernizing clinical processes but also offers a positive return on investment within 18 months, validating its strategic value.

\vspace{6mm}
\noindent\textbf{Keywords:} Hospital Medication Management, Health Information Systems, Patient Safety, \textit{Design Science Research}, Microservices, Interoperability. 