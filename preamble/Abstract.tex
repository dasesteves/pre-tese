\chapter*{Resumo}
\addcontentsline{toc}{chapter}{Resumo}

A gestão medicamentosa em ambiente hospitalar constitui um dos processos mais críticos e complexos das instituições de saúde modernas. O Hospital da Misericórdia de Vila Verde (SCMVV) enfrenta desafios significativos na coordenação entre prescrição médica, validação farmacêutica e administração por enfermagem, operando com sistemas legados fragmentados que comprometem a segurança do paciente e a eficiência operacional.

\vspace{6mm}
\noindent\textbf{Objetivos:} Este projeto visa desenvolver um sistema integrado de gestão medicamentosa que otimize os processos de prescrição, validação, dispensa e administração de medicamentos na SCMVV, garantindo maior segurança do paciente e eficiência operacional através de tecnologias web modernas.

\vspace{6mm}
\noindent\textbf{Metodologia:} O sistema foi desenvolvido seguindo uma arquitetura em camadas baseada em microserviços, utilizando React/Next.js para o frontend, Node.js/Express para o backend, e Oracle Database para persistência de dados. A implementação seguiu metodologia ágil com desenvolvimento incremental e envolvimento contínuo dos utilizadores.

\vspace{6mm}
\noindent\textbf{Resultados:} O sistema demonstrou redução de 73\% nos erros de medicação, melhoria de 80\% nos tempos de resposta, e satisfação dos utilizadores de 8.8/10. A integração com sistemas legados foi bem-sucedida, mantendo compatibilidade total com o AIDA-PCE existente.

\vspace{6mm}
\noindent\textbf{Conclusões:} O projeto demonstrou a viabilidade de modernizar sistemas de gestão medicamentosa em ambiente hospitalar através de tecnologias web modernas, resultando em melhorias significativas na segurança do paciente e eficiência operacional. O ROI positivo projetado em 18 meses justifica o investimento em modernização tecnológica.

\vspace{6mm}
\noindent\textbf{Palavras-chave:} Gestão medicamentosa hospitalar, Sistemas de informação em saúde, Segurança do paciente, Microserviços, React, Node.js, Oracle Database.

\cleardoublepage

\chapter*{Abstract}
\addcontentsline{toc}{chapter}{Abstract}

Hospital medication management constitutes one of the most critical and complex processes in modern healthcare institutions. The Hospital da Misericórdia de Vila Verde (SCMVV) faces significant challenges in coordinating between medical prescription, pharmaceutical validation, and nursing administration, operating with fragmented legacy systems that compromise patient safety and operational efficiency.

\vspace{6mm}
\noindent\textbf{Objectives:} This project aims to develop an integrated medication management system that optimizes prescription, validation, dispensing, and administration processes at SCMVV, ensuring greater patient safety and operational efficiency through modern web technologies.

\vspace{6mm}
\noindent\textbf{Methodology:} The system was developed following a layered architecture based on microservices, using React/Next.js for frontend, Node.js/Express for backend, and Oracle Database for data persistence. Implementation followed agile methodology with incremental development and continuous user involvement.

\vspace{6mm}
\noindent\textbf{Results:} The system demonstrated 73\% reduction in medication errors, 80\% improvement in response times, and user satisfaction of 8.8/10. Integration with legacy systems was successful, maintaining full compatibility with existing AIDA-PCE.

\vspace{6mm}
\noindent\textbf{Conclusions:} The project demonstrated the feasibility of modernizing medication management systems in hospital environments through modern web technologies, resulting in significant improvements in patient safety and operational efficiency. The positive ROI projected in 18 months justifies investment in technological modernization.

\vspace{6mm}
\noindent\textbf{Keywords:} Hospital medication management, Healthcare information systems, Patient safety, Microservices, React, Node.js, Oracle Database. 