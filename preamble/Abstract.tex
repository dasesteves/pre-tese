\chapter*{Resumo}
\addcontentsline{toc}{chapter}{Resumo}

A gestão do ciclo do medicamento em ambiente hospitalar representa um desafio crítico, marcado pela complexidade e pela necessidade de coordenação entre prescrição, validação farmacêutica e administração. Este projeto aborda as ineficiências operacionais e os riscos para a segurança do doente no Hospital da Misericórdia de Vila Verde (SCMVV), decorrentes de sistemas legados fragmentados. Para tal, foi desenvolvido um sistema de informação integrado, com o objetivo de otimizar todo o processo. A solução adota uma arquitetura de microserviços, com um frontend em React/Next.js e um backend em Node.js/Express, sobre uma base de dados Oracle, tendo a sua implementação seguido uma metodologia ágil. Como resultado, o sistema demonstrou uma redução de 73\% nos erros de medicação, uma melhoria de 80\% nos tempos de resposta e uma satisfação dos utilizadores de 8.8/10, garantindo total compatibilidade com os sistemas existentes. Conclui-se que a modernização de processos através de tecnologias web é não só viável, mas também gera melhorias significativas na segurança e eficiência, com um retorno de investimento projetado em 18 meses a justificar a sua implementação.

\vspace{6mm}
\noindent\textbf{Palavras-chave:} Gestão medicamentosa hospitalar, Sistemas de informação em saúde, Segurança do paciente, Microserviços, React, Node.js, Oracle Database.

\vspace*{\fill}

\chapter*{Abstract}
\addcontentsline{toc}{chapter}{Abstract}

Medication management in hospital settings represents a critical challenge, marked by its complexity and the required coordination between prescription, pharmaceutical validation, and administration. This project addresses the operational inefficiencies and patient safety risks at the Hospital da Misericórdia de Vila Verde (SCMVV), which arise from fragmented legacy systems. To this end, an integrated information system was developed to optimize the entire medication lifecycle. The solution adopts a microservices architecture, with a React/Next.js frontend and a Node.js/Express backend, supported by an Oracle Database, and was implemented following an agile methodology. As a result, the system demonstrated a 73\% reduction in medication errors, an 80\% improvement in response times, and a user satisfaction score of 8.8/10, while maintaining full compatibility with existing systems. In conclusion, modernizing processes with web technologies is not only feasible but also yields significant improvements in safety and efficiency, with a projected return on investment in 18 months justifying its implementation.

\vspace{6mm}
\noindent\textbf{Keywords:} Hospital medication management, Healthcare information systems, Patient safety, Microservices, React, Node.js, Oracle Database. 