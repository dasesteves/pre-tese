\chapter{Work Plan and Methodology}

\section{Development Methodology}

This project adopted an agile methodology adapted to the hospital context, combining elements of Scrum and Kanban with specific considerations for critical healthcare systems. This approach was designed to ensure continuous value delivery, flexibility in the face of changing requirements, and close collaboration with healthcare professionals throughout the development lifecycle.

\subsection{Methodological Principles}

The development process was guided by four core principles:

\begin{enumerate}
    \item \textbf{Incremental Development}: The system was built in small, iterative cycles, allowing for frequent deliveries and continuous validation with end-users. This approach minimized risk and ensured the final product was aligned with clinical needs.
    \item \textbf{User Involvement}: Healthcare professionals (physicians, pharmacists, and nurses) were integral members of the development team. Their active participation in all phases, from requirements gathering to testing, was crucial for the project's success.
    \item \textbf{Rapid Prototyping}: Functional prototypes were used extensively to validate concepts and design choices early in the process. This facilitated early feedback and ensured the user interface was intuitive and efficient.
    \item \textbf{Continuous Integration}: Automated testing and controlled deployment pipelines were implemented to maintain code quality, detect regressions early, and ensure system stability.
\end{enumerate}

\section{Project Phases and Timeline}

The project was structured into seven distinct phases, executed over a 12-month period. This phased approach ensured a structured progression from initial analysis to final deployment and evaluation.

\subsection{Phase 1: Analysis and Planning (January-February 2025)}

\textbf{Objectives:}
- Detailed elicitation of functional and non-functional requirements.
- In-depth analysis of the legacy AIDA-PCE system and existing workflows.
- Definition of the high-level technical architecture.

\textbf{Deliverables:}
- Software Requirements Specification (SRS) document.
- AS-IS and TO-BE process mapping diagrams.
- High-level system architecture design document.

\textbf{Key Activities:}
\begin{itemize}
    \item Conducted semi-structured interviews with 15 key stakeholders (5 physicians, 5 nurses, 5 pharmacists).
    \item Performed 40 hours of direct observation of clinical processes.
    \item Analyzed a dataset of 10,000 historical prescriptions to identify patterns and pain points.
    \item Reviewed existing technical documentation for the AIDA system.
\end{itemize}

\subsection{Phase 2: Core Infrastructure Development (March-April 2025)}

\textbf{Objectives:}
- Set up the development, testing, and staging environments.
- Implement the core data access layer and optimize database connections.
- Develop the authentication and authorization system based on JWT.

\textbf{Deliverables:}
- Optimized Oracle database connection pool.
- Secure JWT-based authentication system with role-based access control.
- Base RESTful APIs for core CRUD operations.

\textbf{Performance Metrics:}
- Target average API response time: <200ms.
- Support for 500+ concurrent user sessions.
- Minimum test coverage for core components: 80\%.

\subsection{Phase 3: User Management and Treatment Registration Module (May-June 2025)}

\textbf{Objectives:}
- Develop the user search and patient lookup interface.
- Implement the treatment registration and administration forms.
- Integrate with the hospital's demographic data source.

\textbf{Deliverables:}
- Advanced search component with filtering and sorting capabilities.
- Validated data entry forms with real-time feedback.
- A dynamic dashboard displaying active treatments.

\subsection{Phase 4: Pharmacy and Prescription Validation Module (July-August 2025)}

\textbf{Objectives:}
- Design and implement the prescription validation workflow for pharmacists.
- Develop a real-time inventory management and alerting system.
- Ensure full traceability of medications from dispensing to administration.

\textbf{Deliverables:}
- Pharmaceutical validation interface with integrated clinical decision support.
- Automated low-stock alert system.
- Comprehensive consumption and inventory reports.

\subsection{Phase 5: External System Integrations (September-October 2025)}

\textbf{Integrated Systems:}
\begin{itemize}
    \item \textbf{SONHO}: Data export for billing and administrative purposes.
    \item \textbf{ADSE}: Real-time eligibility verification for insurance coverage.
    \item \textbf{RNU (National User Registry)}: Validation of patient demographic data.
    \item \textbf{PEM (Electronic Medical Prescription)}: Integration with the national e-prescription platform.
\end{itemize}

\textbf{Technical Challenges Addressed:}
- Mapping of disparate data schemas between systems.
- Ensuring real-time data synchronization and consistency.
- Implementing robust error handling and communication failure management.

\subsection{Phase 6: Optimization, Testing, and Validation (November-December 2025)}

\textbf{Activities:}
- Conducted comprehensive load and stress testing to ensure system scalability.
- Performed critical query optimization based on performance profiling.
- Refined user experience (UX) based on feedback from usability testing sessions.
- Executed User Acceptance Testing (UAT) with a cohort of end-users.

\subsection{Phase 7: Documentation and Production Readiness (January 2026)}

\textbf{Final Deliverables:}
- Role-based user manuals for all clinical profiles.
- Complete technical documentation, including API specifications and deployment guides.
- A detailed data migration and system rollout plan.
- Standard Operating Procedures (SOPs) for disaster recovery and business continuity.

\section{Risk Management}

A proactive risk management strategy was employed to identify, assess, and mitigate potential threats to the project's success.

\subsection{Identified Risks and Mitigation Strategies}

\begin{enumerate}
    \item \textbf{Resistance to Change from Staff}
        - \textit{Mitigation:} A comprehensive change management plan was executed, including continuous training, the appointment of departmental "champions" to advocate for the new system, and clear communication about its benefits.
    
    \item \textbf{Technical Incompatibilities with Legacy Systems}
        - \textit{Mitigation:} Extensive integration testing was conducted in a dedicated staging environment that mirrored the production setup. A dedicated team was assigned to resolve compatibility issues.
    
    \item \textbf{System Performance Degradation under Load}
        - \textit{Mitigation:} Proactive performance monitoring was implemented from the early stages. Continuous optimization of database queries, caching strategies, and infrastructure scaling was performed.
    
    \item \textbf{Integration Failures with External Services}
        - \textit{Mitigation:} The system was designed with built-in fault tolerance, including fallback mechanisms and an offline mode for critical functionalities to ensure continuity of care during external service outages.
\end{enumerate}

\section{Resource Allocation}

\subsection{Project Team}
The project was executed by a multidisciplinary team composed of:
- \textbf{Technical Team:} 1 Software Architect (author), 2 Full-Stack Developers (SCMVV collaborators), 1 Oracle DBA (consultant), 1 UX/UI Designer (part-time).
- \textbf{Clinical Team:} 1 Physician (clinical validation lead), 1 Pharmacist (pharmacy requirements lead), 1 Nurse (administration workflow lead).

\subsection{Infrastructure}
- 4 Virtual Machines for development, testing, and staging environments.
- 1 Dedicated Oracle database server.
- All necessary software licenses for development and testing tools.

\section{Monitoring and Control}

\subsection{Key Performance Indicators (KPIs)}

The project's progress and success were tracked using a set of well-defined KPIs:
\begin{itemize}
    \item \textbf{Technical KPIs}: Bug density per sprint, team velocity, and technical debt evolution.
    \item \textbf{Business KPIs}: Reduction in medication error rates, time saved per clinical task, and user adoption rates.
    \item \textbf{Quality KPIs}: Test coverage percentage, code review metrics, and documentation completeness.
\end{itemize}

\subsection{Communication and Governance}

A structured communication plan ensured all stakeholders were kept informed:
- Daily stand-up meetings for the technical team.
- Bi-weekly sprint review and planning sessions with the full project team.
- Monthly steering committee meetings with hospital management.
- Monthly system demonstrations with end-users to gather feedback. 