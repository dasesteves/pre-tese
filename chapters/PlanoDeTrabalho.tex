\chapter{Plano de Trabalho}

% [NOTA: Mover cronograma da introdução para cá e expandir]

\section{Metodologia de Desenvolvimento}

O projeto adotou uma metodologia ágil adaptada ao contexto hospitalar, combinando elementos de Scrum e Kanban com considerações específicas para sistemas críticos de saúde.

\subsection{Princípios Metodológicos}

\begin{enumerate}
    \item \textbf{Desenvolvimento Incremental}: Entregas frequentes para validação contínua
    \item \textbf{Envolvimento dos Utilizadores}: Profissionais de saúde como parte da equipa
    \item \textbf{Prototipagem Rápida}: Validação precoce de conceitos
    \item \textbf{Integração Contínua}: Testes automatizados e deployment controlado
\end{enumerate}

% [INSERIR: Figura 3.1 - Ciclo de desenvolvimento adaptado]

\section{Fases do Projeto}

\subsection{Fase 1: Análise e Planeamento (Janeiro-Fevereiro 2025)}

\textbf{Objetivos:}
- Levantamento detalhado de requisitos
- Análise do sistema legado AIDA-PCE
- Definição da arquitetura técnica

\textbf{Deliverables:}
- Documento de requisitos funcionais e não-funcionais
- Mapeamento de processos AS-IS e TO-BE
- Arquitetura de alto nível

% [INSERIR: Tabela 3.1 - Matriz de requisitos priorizados]

\textbf{Atividades Realizadas:}
\begin{itemize}
    \item Entrevistas com 15 profissionais (5 médicos, 5 enfermeiros, 5 farmacêuticos)
    \item Observação de 40 horas de processos hospitalares
    \item Análise de 10.000 prescrições históricas
    \item Revisão de documentação técnica do AIDA
\end{itemize}

\subsection{Fase 2: Desenvolvimento da Infraestrutura Base (Março-Abril 2025)}

\textbf{Objetivos:}
- Configuração do ambiente de desenvolvimento
- Implementação da camada de dados
- Desenvolvimento do sistema de autenticação

\textbf{Deliverables:}
- Pool de conexões Oracle otimizado
- Sistema JWT com gestão de sessões
- APIs base para CRUD operations

\textbf{Métricas:}
- Tempo de resposta médio: <200ms
- Conexões simultâneas suportadas: 500+
- Cobertura de testes: >80\%

\subsection{Fase 3: Módulo de Procura e Registo (Maio-Junho 2025)}

\textbf{Objetivos:}
- Interface de pesquisa de utentes
- Sistema de registo de tratamentos
- Integração com dados demográficos

% [INSERIR: Figura 3.2 - Mockups das interfaces desenvolvidas]

\textbf{Deliverables:}
- Componente de pesquisa avançada
- Formulários de registo validados
- Dashboard de tratamentos ativos

\subsection{Fase 4: Módulo de Farmácia (Julho-Agosto 2025)}

\textbf{Objetivos:}
- Sistema de validação de prescrições
- Gestão de stocks em tempo real
- Rastreabilidade de medicamentos

\textbf{Deliverables:}
- Interface de validação farmacêutica
- Sistema de alertas de stock
- Relatórios de consumo

% [INSERIR: Tabela 3.2 - Funcionalidades do módulo de farmácia]

\subsection{Fase 5: Integrações Externas (Setembro-Outubro 2025)}

\textbf{Sistemas Integrados:}
\begin{itemize}
    \item \textbf{SONHO}: Exportação para faturação
    \item \textbf{ADSE}: Verificação de elegibilidade
    \item \textbf{RNU}: Validação de dados de utentes
    \item \textbf{PEM}: Prescrição eletrónica nacional
\end{itemize}

\textbf{Desafios Técnicos:}
- Mapeamento de schemas diferentes
- Sincronização de dados em tempo real
- Gestão de falhas de comunicação

\subsection{Fase 6: Otimização e Testes (Novembro-Dezembro 2025)}

\textbf{Atividades:}
- Testes de carga e stress
- Otimização de queries críticas
- Melhorias de UX baseadas em feedback
- Testes de aceitação com utilizadores

% [INSERIR: Gráfico 3.1 - Resultados dos testes de performance]

\subsection{Fase 7: Documentação e Preparação para Produção (Janeiro 2025)}

\textbf{Deliverables Finais:}
- Manual de utilizador por perfil
- Documentação técnica completa
- Plano de migração detalhado
- Procedimentos de disaster recovery

\section{Gestão de Riscos}

% [INSERIR: Tabela 3.3 - Matriz de riscos com probabilidade e impacto]

\subsection{Riscos Identificados e Mitigações}

\begin{enumerate}
    \item \textbf{Resistência à Mudança}
        - Mitigação: Formação contínua e champions internos
    
    \item \textbf{Incompatibilidades Técnicas}
        - Mitigação: Testes extensivos em ambiente de homologação
    
    \item \textbf{Performance Degradada}
        - Mitigação: Monitorização proativa e otimização contínua
    
    \item \textbf{Falhas de Integração}
        - Mitigação: Fallbacks e modo offline
\end{enumerate}

\section{Recursos e Equipa}

\subsection{Equipa Técnica}
- 1 Arquiteto de Software (autor)
- 2 Developers Full-Stack (colaboradores SCMVV)
- 1 DBA Oracle (consultor)
- 1 UX Designer (part-time)

\subsection{Equipa Clínica}
- 1 Médico (validação clínica)
- 1 Farmacêutico (requisitos farmácia)
- 1 Enfermeiro (workflow administração)

\subsection{Infraestrutura}
- 4 VMs para desenvolvimento/teste
- 1 servidor Oracle dedicado
- Licenças de software necessárias

% [INSERIR: Figura 3.3 - Organigrama da equipa do projeto]

\section{Monitorização e Controlo}

\subsection{KPIs do Projeto}

\begin{itemize}
    \item \textbf{Técnicos}: Bugs/sprint, velocidade, debt técnico
    \item \textbf{Negócio}: Redução de erros, tempo poupado, adoção
    \item \textbf{Qualidade}: Cobertura de testes, code reviews, documentação
\end{itemize}

\subsection{Reuniões de Acompanhamento}

- Daily standups (equipa técnica)
- Sprint reviews quinzenais
- Steering committee mensal
- Demos com utilizadores mensais

% [DESENVOLVER: Adicionar Gantt chart detalhado como anexo] 