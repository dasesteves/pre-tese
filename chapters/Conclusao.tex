\chapter{Conclusion and Future Work}

This dissertation detailed the design, implementation, and evaluation of an integrated medication management system aimed at addressing critical patient safety and workflow efficiency challenges within a hospital setting. This final chapter synthesizes the research, reiterates the principal contributions, outlines a strategic roadmap for future work, and offers concluding remarks on the project's broader significance.

\section{Synthesis and Principal Contributions}

This research successfully demonstrated that the strategic application of modern web technologies, combined with a user-centered co-design philosophy, can overcome the fragmentation of legacy hospital information systems. The sociotechnical intervention at SCMVV resulted in a cohesive, integrated medication management workflow, yielding a 73% reduction in medication errors and an 80% improvement in key system response times.

The project delivers several key contributions to the field of Health Informatics. It proposes and validates a \textit{novel integration framework} for modernizing entrenched legacy systems, providing a replicable model for other institutions. It also puts forward a \textit{microservices-based reference architecture} that serves as a scalable and resilient blueprint for future clinical applications \cite{newman2021}. Furthermore, this work documents and validates an \textit{agile implementation methodology} tailored for the complexities of a live hospital environment \cite{may2013}, and proposes a \textit{domain-specific evaluation toolkit} of KPIs to measure the multifaceted impact of such systems \cite{donabedian1988}.

\section{Future Work and Research Agenda}

The completion of this project establishes a foundation for a long-term research and development agenda aimed at creating a more intelligent and interoperable healthcare ecosystem.

The immediate technological roadmap is focused on enhancing the system's intelligence and connectivity. This includes integrating predictive analytics with AI to move from a reactive to a proactive safety model, identifying potential adverse drug events before they occur \cite{bates2021,zhao2021}. A subsequent priority is the development of a mobile-first bedside application to support medication administration at the point of care. Strategically, achieving full conformance with the HL7 FHIR standard is a key goal to ensure seamless, standards-based interoperability with national and international health data ecosystems \cite{mandl2020}.

This work also opens several new avenues for formal academic inquiry. A longitudinal impact assessment is required to understand the long-term effects of the system on patient outcomes and organizational culture \cite{greenhalgh2017}. A multi-center generalizability study would be invaluable to validate the intervention's effectiveness across different institutional contexts. Furthermore, research into the cognitive ergonomics of the user interface could yield new insights into minimizing cognitive load and reducing the risk of technology-induced errors \cite{holden2011}.

\section{Final Remarks}

The digital transformation of healthcare is fundamentally a sociotechnical challenge, demanding a synthesis of technological innovation and a deep understanding of human and organizational factors. The success of this project validates the proposition that a user-centered, agile, and methodologically rigorous approach can successfully modernize critical clinical systems. The system developed herein is more than a technical artifact; it represents a new operational paradigm for medication management, one that is aligned with international best practices and poised to meet the future challenges of digital health. This journey can serve as a valuable case study for other healthcare institutions, demonstrating that such modernization is not only achievable but essential for delivering safe, efficient, and patient-centered care in the 21st century. 