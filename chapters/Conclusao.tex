\chapter{Conclusion and Future Work}

This dissertation addressed the critical challenges of medication management in a hospital setting by designing, developing, and evaluating an integrated software system. This concluding chapter synthesizes the work accomplished, reiterates the principal scientific contributions, outlines promising directions for future work, and offers final considerations on the broader impact of this research.

\section{Synthesis of Accomplished Work}

This project has successfully demonstrated the feasibility and effectiveness of modernizing hospital medication management systems using contemporary web technologies. The system developed for SCMVV successfully integrated previously fragmented processes, leading to significant, quantifiable improvements in patient safety and operational efficiency \cite{ciapponi2021}.

The main achievements of this work include:
\begin{itemize}
    \item A 73% reduction in medication errors, directly enhancing patient safety \cite{radley2013}.
    \item An 80% improvement in system response times, leading to a better user experience.
    \item A high user satisfaction score of 8.8 out of 10, indicating strong user acceptance \cite{venkatesh2003}.
    \item A projected positive Return on Investment (ROI) within 18 months, confirming the economic viability of the solution \cite{adler2021}.
\end{itemize}

\section{Scientific and Technological Contributions}

This research contributes to the body of knowledge in Health Informatics through the following key outputs:

\begin{enumerate}
    \item \textbf{A Replicable Integration Framework:} This work provides a proven, non-invasive model for integrating modern web applications with legacy healthcare systems, a common challenge in the field \cite{keasberry2017}.
    \item \textbf{A Microservices-Based Reference Architecture:} It presents a reference architecture and design patterns for hospital information systems based on a microservices paradigm, offering a blueprint for scalable and resilient clinical applications \cite{newman2021}.
    \item \textbf{An Evidence-Based Implementation Methodology:} The project documents a successful, agile-based process for digital transformation in public hospitals, which can guide similar initiatives in other institutions \cite{may2013}.
    \item \textbf{A Set of Domain-Specific Evaluation Metrics:} It defines and validates a set of Key Performance Indicators (KPIs) tailored specifically for evaluating the impact of hospital medication management systems, building on Donabedian's model \cite{donabedian1988}.
\end{enumerate}

\section{Future Work}

While this project achieved its primary objectives, it also opened up several avenues for future research and development.

\subsection{Technical Enhancements}

\begin{itemize}
    \item \textbf{Artificial Intelligence Integration}: Implement predictive models using machine learning to provide early warnings for potential adverse drug events and complex drug interactions \cite{bates2021,zhao2021}.
    \item \textbf{Native Mobile Application}: Develop a "mobile-first" native application to support medication administration at the patient's bedside, improving workflow efficiency and reducing errors.
    \item \textbf{Full HL7 FHIR Compliance}: Extend the system's integration layer to achieve full compliance with the HL7 FHIR standard, enabling seamless interoperability on a national and international scale \cite{mandl2020}.
    \item \textbf{Blockchain for Traceability}: Explore the use of blockchain technology to create an immutable and fully transparent audit trail for high-risk medications, enhancing supply chain security \cite{franzoso2014}.
\end{itemize}

\subsection{Functional Expansion}

\begin{itemize}
    \item \textbf{Intelligent Inventory Management}: Integrate machine learning algorithms to automate stock level forecasting and optimize pharmacy inventory management \cite{rozenblum2020}.
    \item \textbf{Regional Health Network Integration}: Expand the system to connect with other hospitals and primary care centers in the region, creating a unified patient medication record.
    \item \textbf{Telemedicine and Remote Prescribing}: Add support for telemedicine workflows, enabling secure remote prescribing and patient consultations.
    \item \textbf{Advanced Predictive Analytics}: Develop advanced analytics dashboards for hospital management, providing predictive insights into medication usage patterns, cost trends, and quality indicators \cite{berwick2008}.
\end{itemize}

\subsection{Future Research Directions}

\begin{enumerate}
    \item \textbf{Longitudinal Impact Study}: Conduct a long-term study (e.g., 3-5 years) to evaluate the sustained impact of the system on patient outcomes, organizational culture, and economic performance \cite{greenhalgh2017}.
    \item \textbf{Multi-Center Replication Study}: Replicate the implementation and evaluation of the system in different hospital settings (e.g., large academic medical centers, specialized clinics) to validate the generalizability of the findings.
    \item \textbf{Human Factors and Ergonomics Research}: Conduct an in-depth investigation into the human factors and cognitive ergonomics of the system's interface to further optimize user interaction and minimize cognitive load \cite{holden2011}.
    \item \textbf{Comprehensive Cost-Effectiveness Analysis}: Perform a detailed health economics analysis to quantify the system's cost-effectiveness in terms of quality-adjusted life years (QALYs) gained and other economic metrics \cite{adler2021}.
\end{enumerate}

\section{Final Considerations}

The digital transformation of healthcare is not merely a technological endeavor; it is a fundamental organizational and cultural shift \cite{greenhalgh2017}. The success of this project demonstrates that with the right combination of executive commitment, adequate resources, and a user-centered approach, it is possible to modernize critical health systems, maintain safety, and significantly improve outcomes.

The system developed in this dissertation represents not just a technical solution, but a new paradigm for hospital medication management in Portugal, aligned with international best practices \cite{who2022} and prepared for the future challenges of digital health. The journey of SCMVV serves as a model and an inspiration for other healthcare institutions facing similar challenges, proving that modernization is not only possible but essential for delivering safe, efficient, and patient-centered care in the 21st century. 