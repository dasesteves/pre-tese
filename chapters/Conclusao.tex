\chapter{Conclusion and Future Work}
\label{chap:Conclusion}

This dissertation proposal has outlined the design, development, and evaluation plan for an integrated medication management system aimed at addressing critical patient safety and workflow efficiency challenges within a hospital setting. This final chapter synthesizes the proposed research, reiterates its potential contributions, outlines a strategic roadmap for future work, and offers concluding remarks on the project's broader significance.

\section{Synthesis and Potential Contributions}

This research aims to demonstrate that the strategic application of modern web technologies, combined with a user-centered co-design philosophy, can overcome the fragmentation endemic to legacy hospital information systems. The proposed sociotechnical intervention at SCMVV is designed to create a cohesive, integrated medication management workflow, with the anticipated outcomes of significantly reducing medication errors and improving key system response times.

If successful, this project is expected to deliver several key contributions to the field of Health Informatics. It will propose and validate a \textit{novel integration framework} for modernizing entrenched legacy systems, providing a replicable model for other institutions. It will also put forward a \textit{microservices-based reference architecture} intended to serve as a scalable and resilient blueprint for future clinical applications \cite{newman2021}. Furthermore, this work will document and validate an \textit{agile implementation methodology} tailored for the complexities of a live hospital environment \cite{may2013}, and will propose a \textit{domain-specific evaluation toolkit} of KPIs to measure the multifaceted impact of such systems \cite{donabedian1988}.

\section{Future Work and Research Agenda}

The completion of this project will establish a robust foundation for a long-term research and development agenda aimed at creating a more intelligent and interoperable healthcare ecosystem.

The immediate technological roadmap following this work will focus on enhancing the system's intelligence and connectivity. This includes integrating predictive analytics with AI to move from a reactive to a proactive safety model, identifying potential adverse drug events before they occur \cite{bates2021,zhao2021}. A subsequent priority will be the development of a mobile-first bedside application to support medication administration at the point of care. Strategically, achieving full conformance with the HL7 FHIR standard is a key future goal to ensure seamless, standards-based interoperability with national and international health data ecosystems \cite{mandl2020}.

This work will also open several new avenues for formal academic inquiry. A longitudinal impact assessment will be required to understand the long-term effects of the system on patient outcomes and organizational culture \cite{greenhalgh2017}. A multi-center generalizability study would be invaluable to validate the intervention's effectiveness across different institutional contexts. Furthermore, research into the cognitive ergonomics of the user interface could yield new insights into minimizing cognitive load and reducing the risk of technology-induced errors \cite{holden2011}.

\section{Final Remarks}

The digital transformation of healthcare is fundamentally a sociotechnical challenge, demanding a synthesis of technological innovation and a deep understanding of human and organizational factors. This proposed project is built on the proposition that a user-centered, agile, and methodologically rigorous approach can successfully modernize critical clinical systems. The system to be developed is more than a technical artifact; it represents a new operational paradigm for medication management, one that is aligned with international best practices and poised to meet the future challenges of digital health. This journey can serve as a valuable case study for other healthcare institutions, demonstrating that such modernization is not only achievable but essential for delivering safe, efficient, and patient-centered care in the 21st century. 