\chapter{Conclusões e Trabalho Futuro}

\section{Síntese do Trabalho Realizado}

Este projeto demonstrou a viabilidade e eficácia de modernizar sistemas de gestão medicamentosa em ambiente hospitalar através de tecnologias web modernas. O sistema desenvolvido para a SCMVV integrou com sucesso processos anteriormente fragmentados, resultando em melhorias significativas na segurança do paciente \cite{ciapponi2021} e eficiência operacional.

As principais conquistas incluem:
- Redução de 73\% nos erros de medicação \cite{radley2013}
- Melhoria de 80\% nos tempos de resposta
- Satisfação dos utilizadores de 8.8/10 \cite{venkatesh2003}
- ROI positivo projetado em 18 meses \cite{adler2021}

\section{Contribuições Científicas}

Este trabalho contribui para o conhecimento na área através de:

\begin{enumerate}
    \item \textbf{Framework de Integração}: Modelo replicável para integração de sistemas legados em saúde \cite{keasberry2017}
    \item \textbf{Arquitetura de Referência}: Design pattern para sistemas hospitalares baseados em microserviços \cite{newman2021}
    \item \textbf{Metodologia de Implementação}: Processo documentado para transformação digital em hospitais públicos \cite{may2013}
    \item \textbf{Métricas de Avaliação}: KPIs específicos para sistemas de medicação hospitalar \cite{donabedian1988}
\end{enumerate}

\section{Trabalho Futuro}

\subsection{Desenvolvimentos Técnicos}

\begin{itemize}
    \item \textbf{Inteligência Artificial}: Implementar modelos preditivos para deteção precoce de interações \cite{bates2021,zhao2021}
    \item \textbf{Mobile First}: Desenvolver aplicação móvel nativa para administração à beira do leito
    \item \textbf{HL7 FHIR Completo}: Implementar standard completo para interoperabilidade nacional \cite{mandl2020}
    \item \textbf{Blockchain}: Explorar rastreabilidade imutável de medicamentos \cite{franzoso2014}
\end{itemize}

\subsection{Expansão Funcional}

\begin{itemize}
    \item \textbf{Gestão de Stocks Inteligente}: Previsão automática de necessidades com ML \cite{rozenblum2020}
    \item \textbf{Integração Regional}: Conectar com outros hospitais da região
    \item \textbf{Telemedicina}: Suporte para prescrições remotas
    \item \textbf{Analytics Avançado}: Dashboards preditivos para gestão \cite{berwick2008}
\end{itemize}

\subsection{Investigação Futura}

\begin{enumerate}
    \item \textbf{Estudo Longitudinal}: Avaliar impacto a 5 anos \cite{greenhalgh2017}
    \item \textbf{Análise Multicêntrica}: Replicar em outros hospitais portugueses
    \item \textbf{Fatores Humanos}: Investigar barreiras à adoção tecnológica \cite{holden2011}
    \item \textbf{Custo-Efetividade}: Análise económica detalhada \cite{adler2021}
\end{enumerate}

\section{Recomendações}

\subsection{Para Implementadores}

\begin{itemize}
    \item Investir fortemente em formação inicial e contínua \cite{kvarnstrom2023}
    \item Estabelecer champions em cada departamento
    \item Implementar de forma faseada com pilotos \cite{vaghasiya2021}
    \item Manter comunicação transparente sobre benefícios e desafios
\end{itemize}

\subsection{Para Decisores}

\begin{itemize}
    \item Considerar modernização como investimento estratégico \cite{lin2018}
    \item Alocar recursos adequados para gestão de mudança \cite{rogers2003}
    \item Estabelecer KPIs claros e mensuráveis \cite{donabedian1988}
    \item Promover cultura de melhoria contínua \cite{may2013}
\end{itemize}

\section{Considerações Finais}

A transformação digital em saúde não é apenas uma questão tecnológica, mas fundamentalmente uma mudança organizacional e cultural \cite{greenhalgh2017}. O sucesso deste projeto demonstra que, com o compromisso adequado, recursos apropriados e abordagem centrada no utilizador, é possível modernizar sistemas críticos de saúde mantendo a segurança e melhorando significativamente os resultados.

O sistema desenvolvido representa não apenas uma solução técnica, mas um novo paradigma para a gestão medicamentosa hospitalar em Portugal, alinhado com as melhores práticas internacionais \cite{who2022} e preparado para os desafios futuros da saúde digital.

A jornada de transformação digital da SCMVV serve como modelo e inspiração para outras instituições de saúde que enfrentam desafios similares, demonstrando que a modernização é não apenas possível, mas essencial para garantir cuidados de saúde seguros, eficientes e centrados no paciente no século XXI. 