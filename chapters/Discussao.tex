\chapter{Discussão}

\section{Análise dos Resultados}

\subsection{Sucesso da Implementação}

Os resultados obtidos demonstram claramente que o sistema desenvolvido atingiu os objetivos propostos, superando as expectativas iniciais em múltiplas dimensões. A redução de 73% nos erros de medicação \cite{ciapponi2021,radley2013} representa um impacto significativo na segurança do paciente, alinhando-se com os melhores resultados reportados na literatura internacional.

A melhoria substancial nos tempos de resposta, superior a 80%, traduziu-se numa experiência de utilizador notavelmente melhorada. A alta satisfação dos utilizadores, com uma classificação média de 8.8/10 \cite{holden2011}, reflete a aceitação positiva da solução pelos profissionais de saúde. O retorno do investimento positivo em 18 meses \cite{adler2021} valida economicamente a implementação e justifica investimentos futuros em tecnologias similares.

\subsection{Fatores Críticos de Sucesso}

A análise dos resultados permite identificar quatro fatores críticos que contribuíram decisivamente para o sucesso da implementação. O envolvimento ativo dos utilizadores durante todo o processo de desenvolvimento \cite{venkatesh2003} foi fundamental, permitindo um desenvolvimento iterativo com feedback contínuo que assegurou que o sistema atendesse às necessidades reais dos profissionais de saúde.

A arquitetura flexível baseada em microserviços \cite{newman2021} revelou-se crucial, permitindo uma evolução gradual do sistema sem interrupções significativas dos serviços hospitalares. A formação adequada dos utilizadores, com 40 horas de formação por utilizador \cite{kvarnstrom2023}, garantiu uma adoção eficaz e reduziu a resistência à mudança. Finalmente, o compromisso e suporte executivo da administração hospitalar providenciaram os recursos necessários e legitimaram a mudança organizacional.

\section{Desafios Encontrados}

\subsection{Desafios Técnicos}

\begin{itemize}
    \item \textbf{Integração com Sistemas Legados}: Complexidade do AIDA-PCE exigiu engenharia reversa \cite{keasberry2017}
    \item \textbf{Performance Oracle}: Otimização de queries para grandes volumes \cite{jiang2014}
    \item \textbf{Sincronização de Dados}: Garantir consistência entre sistemas
    \item \textbf{Gestão de Sessões}: Implementação de SSO complexa
\end{itemize}

\subsection{Desafios Organizacionais}

\begin{itemize}
    \item \textbf{Resistência à Mudança}: 30\% dos utilizadores inicialmente relutantes \cite{rogers2003}
    \item \textbf{Processos Enraizados}: Dificuldade em alterar workflows de 20+ anos
    \item \textbf{Coordenação Interdepartamental}: Alinhamento entre TI, farmácia e clínica
    \item \textbf{Gestão de Expectativas}: Pressão por resultados imediatos
\end{itemize}

\section{Lições Aprendidas}

\subsection{Aspetos Positivos}

\begin{enumerate}
    \item \textbf{Abordagem Incremental}: Implementação faseada reduziu riscos \cite{may2013}
    \item \textbf{Prototipagem Rápida}: Validação precoce de conceitos
    \item \textbf{Documentação Extensiva}: Facilitou manutenção e onboarding
    \item \textbf{Testes Automatizados}: Deteção precoce de regressões \cite{fowler2018}
\end{enumerate}

\subsection{Áreas de Melhoria}

\begin{enumerate}
    \item \textbf{Gestão de Estado}: Context API mostrou limitações em componentes complexos
    \item \textbf{Performance Mobile}: Necessidade de otimizações adicionais
    \item \textbf{Monitorização}: Implementação tardia dificultou diagnósticos iniciais
    \item \textbf{Gestão de Dependências}: Atualizações de segurança complexas
\end{enumerate}

\section{Comparação com Literatura}

Os resultados obtidos alinham-se com estudos internacionais:
- Redução de erros (73\%) consistente com meta-análise de Radley et al. \cite{radley2013} (81%)
- Satisfação dos utilizadores (8.8/10) superior à média reportada \cite{hertzum2022} (7.2/10)
- ROI (18 meses) mais rápido que média hospitalar \cite{adler2021} (24-36 meses)
- Taxa de adoção (87\%) acima do esperado segundo modelo TAM \cite{venkatesh2003} (65-70%)

\section{Limitações do Estudo}

\subsection{Limitações Metodológicas}

\begin{itemize}
    \item \textbf{Período de Avaliação}: 6 meses pode ser insuficiente para efeitos a longo prazo \cite{greenhalgh2017}
    \item \textbf{Contexto Único}: Resultados específicos da SCMVV
    \item \textbf{Ausência de Grupo Controlo}: Comparação antes/depois tem limitações
    \item \textbf{Viés de Seleção}: Utilizadores mais motivados podem ter participado mais
\end{itemize}

\subsection{Limitações Técnicas}

\begin{itemize}
    \item \textbf{Dependência do Oracle}: Vendor lock-in potencial \cite{lin2018}
    \item \textbf{Escalabilidade Horizontal}: Ainda não testada completamente
    \item \textbf{Integração HL7 FHIR}: Implementação parcial \cite{mandl2020}
    \item \textbf{Machine Learning}: Funcionalidades preditivas não implementadas \cite{bates2021}
\end{itemize}

\section{Implicações Práticas}

\subsection{Para a Prática Clínica}

- Demonstra viabilidade de modernização em hospitais públicos
- Confirma importância da usabilidade na adoção \cite{mcgreevey2020}
- Valida abordagem de desenvolvimento ágil em saúde \cite{vaghasiya2021}
- Reforça necessidade de formação contínua \cite{kvarnstrom2023}

\subsection{Para Gestores Hospitalares}

- ROI justifica investimento em modernização tecnológica \cite{rozenblum2020}
- Importância do suporte executivo para sucesso
- Necessidade de gestão de mudança estruturada \cite{may2013}
- Valor da monitorização contínua de KPIs \cite{donabedian1988} 