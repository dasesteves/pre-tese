\chapter{Discussion}
\label{chap:Discussion}

This chapter provides a prospective analysis of the expected outcomes of this research, contextualizing their potential significance within the existing body of scientific literature and the specific operational realities of the Portuguese National Health Service (SNS). It will critically examine the anticipated implications of the key findings, the foreseeable challenges of implementation, and the inherent limitations of the study's design. The chapter will conclude by outlining the broader implications of this work for clinical practice, hospital management, and future research in healthcare informatics.

\section{Interpretation of Expected Implications}

The central thesis of this work is that a strategically designed, unified frontend architecture can serve as a powerful catalyst for overcoming systemic fragmentation in hospital information systems. We anticipate that the results will demonstrate a statistically significant reduction in medication errors and a tangible improvement in clinical workflow efficiency. However, the interpretation of these findings will transcend the raw metrics. The expected 73\% reduction in medication errors, for instance, should be interpreted not merely as a technical achievement but as a validation of \textit{user-centered design principles} in mitigating clinical risk \cite{ciapponi2021,radley2013}.

Similarly, the projected improvements in system performance and user satisfaction are expected to provide evidence for the thesis that modernizing the user-facing layer of technology can yield disproportionately high returns, even when legacy backend systems remain partially in place. This suggests a crucial strategic lesson for hospital administrators: high-impact modernization does not always require a complete, high-risk "rip-and-replace" overhaul of the entire infrastructure \cite{adler2021}. The success of the microservices-based architecture is expected to reinforce the value of architectural flexibility and incremental deployment in complex, risk-averse environments \cite{newman2021}.

\section{Anticipated Challenges and Contextualization}

The successful implementation of this project hinges on navigating significant sociotechnical challenges, particularly within the high-pressure context of the Portuguese public healthcare system \cite{goiana2024portuguese}. While the technical hurdles of integrating with legacy systems are considerable \cite{keasberry2017}, the primary challenges are anticipated to be human and organizational. Introducing a new system to clinical staff already facing significant workload pressures requires a change management strategy that is empathetic, inclusive, and demonstrates immediate value \cite{rogers2003}.

The project's success will therefore depend on the effective application of the user-centered co-design philosophy, ensuring clinicians are not just subjects of the change, but active partners in its design and rollout \cite{venkatesh2003}. We anticipate encountering resistance rooted in established workflows and cognitive fatigue. The mitigation strategy relies on an agile, iterative implementation that allows for rapid feedback and adjustment, empowering clinical champions to advocate for the system and demonstrating tangible workflow improvements from the earliest stages \cite{may2013}. This approach directly confronts the problem of systemic fragmentation observed in the national context, where a lack of integration forces clinicians to become "human middleware," bridging information gaps between disparate systems \cite{pinto2016identification}.

\section{Limitations and Avenues for Future Research}

The findings of this study must be interpreted within the boundaries of its methodological design, which present clear avenues for future research. The single-center design, while necessary for a deep, context-specific implementation at SCMVV, inherently limits the statistical generalizability of the findings to other institutions with different organizational cultures or technical infrastructures. The quasi-experimental design, lacking a parallel control group, means that while we can measure significant improvements, we cannot definitively exclude the influence of confounding variables.

Furthermore, the study's evaluation will focus on objective metrics of patient safety and operational efficiency. It is acknowledged that the implementation of new information systems has a profound impact on the psychosocial dimensions of work, including the cognitive load and potential for burnout among healthcare professionals \cite{hertzum2022}. A detailed analysis of these factors, while critically important, falls outside the defined scope of this dissertation and represents a significant and necessary direction for future investigation.

Technically, while the proposed architecture promotes interoperability, this initial phase will not achieve full conformance with standards such as HL7 FHIR. Achieving this level of semantic interoperability is a crucial next step, paving the way for seamless data exchange with national health platforms and other providers \cite{mandl2020}.

Despite these limitations, this work is poised to make significant contributions. For clinical practice, it will offer a validated model for modernizing critical hospital workflows. For management, it will present a data-driven case for investing in user-experience-focused technology. For research, it will lay the groundwork for future studies on long-term impacts, scalability, and the broader effects of technological change on the healthcare workforce. 